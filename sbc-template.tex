\documentclass[12pt]{article}

\usepackage{cmap}
\usepackage{sbc-template}
\usepackage{listings}
\usepackage{graphicx,url}
\usepackage[T1]{fontenc}
\usepackage{amsmath,amssymb}%,amsthm}

\usepackage[brazil]{babel}
\usepackage[utf8]{inputenc}

\title{Protocolo RDN}

\author{
  Jonathas Augusto de Oliveira Conceição\inst{1}
}

\address{
  Curso de Bacharelado em Ciêncai da Computação\\
  Universidade Federal de Pelotas (UFPel)
  \email{jadoliveira@inf.ufpel.edu.br}
}

\begin{document}

\maketitle

\begin{resumo} %% TODO: resumir protocolo
  Resumo do protocolo
\end{resumo}

\section{Introdução}

<Visão geral do protocolo>

O artigo é organizado da seguinte forma:
A Seção \ref{sec:MaquinaDeEstados} apresenta as máquinas de estados do cliente emissor, cliente receptor e do servidor;
na Seção \ref{sec:Mensagens} todos os tipos de mensagens e seus cabeçalhos são apresentados;
por fim, a Seção \ref{sec:Conclusao} conclui apresentado alguns casos de uso do protocolo.

\section{Máquina de Estados}\label{sec:MaquinaDeEstados}
<Introdução máquinas>
\subsection{Cliente Emissor}\label{sec:Emissor}
<Breve descrição das funcionalidades do cliente emissor>
%% \begin{figure}%[ht]
%%   \centering
%%   \lstinputlisting[frame=lines]{images/asd.png}
%%   \caption{Legenda.}
%%   \label{}
%% \end{figure}
\subsection{Cliente Receptor}\label{sec:Receptor}
<Breve descrição das funcionalidades do cliente receptor>
\subsection{Servidor}\label{sec:Servidor}
<Breve descrição das funcionalidades do servidor>
\section{Mensagens}\label{sec:Mensagens}
<Introdução mensagens>
\section{Conclusão}\label{sec:Conclusao}
<Exemplos de uso e conclusão>

\end{document}
