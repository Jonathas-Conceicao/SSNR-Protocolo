\documentclass[12pt]{article}

\usepackage{cmap}
\usepackage{sbc-template}
\usepackage{listings}
\usepackage{graphicx,url}
\usepackage[T1]{fontenc}
\usepackage{amsmath,amssymb}

\usepackage[brazil]{babel}
\usepackage[utf8]{inputenc}

\title{Protocolo Simple Remote Notifcation System (SRN)}

\author{
	Jonathas Augusto de Oliveira Conceição\inst{1}
}

\address{
	Curso de Bacharelado em Ciêncai da Computação\\
	Universidade Federal de Pelotas (UFPel)
	\email{jadoliveira@inf.ufpel.edu.br}
}

\begin{document}

\maketitle

\begin{resumo}
	O Simple Remote Notification System (Sistema Simples de Notificação Remota, ou apenas SRN)
	é um protocolo de notificações distribuída em modelo cliente-servidor.
	Nele há dois tipos de possíveis clientes, emissores e receptores de notificações.
	No sistema os clientes se comunicam por meio de um servidor que contem a identificação de casa usuário e,
	sempre que recebe uma notificação,
	envia ao destinatário correspondente.
	Um cliente pode ser apenas emissor ou apenas receptor, simplificando assim a implementação do protocolo em um dispositivo.
\end{resumo}

%%=========================================================================================
\section{Introdução}
%%=========================================================================================

O Simple Remote Notification System (Sistema Simples de Notificação Remota, ou apenas SRN)
é um protocolo de notificações distribuída em modelo cliente-servidor.
No sistema a comunicação é feita por meio de um servidor que contem a identificação de receptores,
e por sua vez, repassa a notificação ao destinatário correspondente.

O protocolo define dois tipos de clientes, emissores e receptores de notificações.
Receptores podem se conectar à um servidor enviando uma mensagem de conexão com o nome desejado.
Uma vez conetados eles passam a receber notificações de qualquer remetente com acesso aquele servidor que deseje contacta-los.
Emissores podem enviar mensagens ao servidor requisitando uma lista de clientes conectados,
e podem também enviar notificações à clientes conectados ao servidor.

O servidor então tem três funcionalidades principais:
(1) indexar receptores conectados;
(2) enviar lista dos receptores disponíveis à emissores;
(3) receber e entregar notificações vinda de um emissor para um receptor válido.

O artigo é organizado da seguinte forma:
A Seção \ref{sec:MaquinaDeEstados} apresenta as máquinas de estados do cliente emissor, cliente receptor e do servidor;
na Seção \ref{sec:Mensagens} todos os tipos de mensagens e seus cabeçalhos são apresentados;
por fim, a Seção \ref{sec:Conclusao} conclui apresentado alguns casos de uso do protocolo.

%%=========================================================================================
\section{Máquina de Estados}\label{sec:MaquinaDeEstados}
%%=========================================================================================

Nesta seção são apresentados as máquinas de estado para o
Cliente Emissor,
o Cliente Receptor e
o Servidor.
Os círculos representam estados do sistema e
as transações são dadas por dois componentes:
um evento que pode acontecer;
e uma ação a ser tomada.
Os componentes da transição são separados por um traço vertical,
o evento sempre se encontra acima do traço e
a ação a ser tomada se encontra abaixo.

\subsection{Cliente Emissor}\label{sec:Emissor}
<Breve descrição das funcionalidades do cliente emissor>
%% \begin{figure}%[ht]
%%   \centering
%%   \lstinputlisting[frame=lines]{images/asd.png}
%%   \caption{Legenda.}
%%   \label{}
%% \end{figure}
\subsection{Cliente Receptor}\label{sec:Receptor}
<Breve descrição das funcionalidades do cliente receptor>
\subsection{Servidor}\label{sec:Servidor}
<Breve descrição das funcionalidades do servidor>

%%=========================================================================================
\section{Mensagens}\label{sec:Mensagens}
%%=========================================================================================
<Introdução mensagens>

%%=========================================================================================
\section{Conclusão}\label{sec:Conclusao}
%%=========================================================================================
<Exemplos de uso e conclusão>

\end{document}
